\section{Cahier des charges}
	\par
	Le cahier des charges a été formulé comme il suit : L'objectif du
	projet consiste en la conception et la réalisation de capteurs autonomes
	communiquant en WiFi afin de pouvoir remonter régulièrement des informations
	sur l'état des salles de cours. Les capteurs seront par exemple des détecteurs de
	lumières, de pression, de qualité de l'air, ...
	\par
	La communication sera obligatoirement réalisée en WiFi sur le réseau
	de l'université en respectant les contraintes de sécurité (WPA2, ...).
	\par
	Deux options sont possibles :
	\begin{itemize}
    \item Refaire complètement une carte avec un microcontroleur et une
    puce wifi (comme par exemple les \href{https://www.spark.io/}{spark})
    \item Réaliser un shield pour raspberry pi contenant les différents capteurs. 
	\end{itemize}
	
\section{Présentation de la réalisation}
	\subsection{Présentation de la partie web}
	
	\subsection{Présentation de la partie capteur}
		\subsubsection{Le hardware}
		\par
		A la fin du projet, nous avons réalisé un prototype fonctionnel de ce
		que pourrait être le capteur. Il est important de noter que nous n'avons
		pas cherché à intégrer la solution mais plutôt à la développer en terme
		de fonctionnalités. De ce fait le prototype est assez volumineux mais
		sa taille peut aisément être réduite en utilisant des solutions à base
		de composants CMS notemment ce qui permettrait de réduire de manière
		significative l'encombrement de l'appareil.
		\par
		Le prototype final se compose donc de trois éléments principaux. Le coeur
		du système est réalisé avec un Arduino Uno. Une platine de développement
		sur microcontrôleur basée sur un ATmega328P de chez Atmel. C'est dans cet
		Arduino qu'est flashé le programme du capteur. Vient en suite la partie
		qui prend en charge le WiFi, il s'agit d'un shield WiFi de chez GoTronic
		basée sur la puce WizFi210 de WizNet, cette dernière est interfacée avec
		l'Arduino via le port série. Enfin la dernière partie est un protoshield
		sur lequel est placé une breadboard sur laquelle sont enfichés les composants
		avec lesquels l'Arduino peut intéragir, à savoir :
		\begin{itemize}
			\item Une photorésistance pour capter la luminosité
			\item Un capteur TMP36 de température
			\item Une RTC (Real Time Clock) DS1302
		\end{itemize}
		Notons que nous avons choisi ces capteurs pour leur facilité d'utilisation,
		mais dans l'absolu rien n'empêche d'utiliser n'importe quel autre capteur
		analogique ou TOR, ni même d'interfacer le système avec des choses plus
		complexes à base de SPI ou d'I$^2$C, étant donné que l'ATmega328P en a
		la possibilité.
		\par
		Le principe de fonctionnement du capteur est très simple. Une fois configuré
		à l'aide de \texttt{configurator} (dont le fonctionnement sera détaillé ci après)
		il vous suffit de déployer le capteur là où vous souhaitez qu'il opère et le
		laisser envoyer des mesures à interval d'une minute. On dispose ainsi
		d'un suivi efficace de l'environnement. Aussitôt que l'appareil est
		mis sous tension il va essayer de se connecter au réseau qui lui a été
		spécifié en configuration et une fois que cela s'est fait proprement il
		va passer en mode interruptif et enverra ses mises à jour toutes les minutes.
		Il faut aussi noter que pendant ce temps là, le capteur écoutera les connexions
		entrantes sur le port 80 et pourra ainsi être reconfiguré par ce biais à l'aide
		d'un set de commandes AT décris et documenté en annexe.
		
		\subsubsection{Le software}
