% !TEX encoding = UTF-8 Unicode
\section{Cahier des charges}
	\par
	Le cahier des charges a été formulé comme il suit : L'objectif du
	projet consiste en la conception et la réalisation de capteurs autonomes
	communiquant en WiFi afin de pouvoir remonter réguliérement des informations
	sur l'état des salles de cours. Les capteurs seront par exemple des détecteurs de
	lumières, de pression, de qualité de l'air, ...
	\par
	La communication sera obligatoirement réalisée en WiFi sur le réseau
	de l'université en respectant les contraintes de sécurité (WPA2, ...).
	\par
	Deux options sont possibles :
	\begin{itemize}
    \item Refaire complètement une carte avec un microcontroleur et une
    puce wifi (comme par exemple les \href{https://www.spark.io/}{spark})
    \item Réaliser un shield pour raspberry pi contenant les différents capteurs. 
	\end{itemize}
	
\section{Présentation de la réalisation}
	\subsection{Présentation de la partie web}
		\par
		La partie Web du projet offre une interface de supervision et de monitoring des capteurs. En effet, nous avons pense qu'il ne suffisait pas de réussir à envoyer des données en Wifi par le biais des capteurs, il fallait aussi pouvoir récupérer et stocker ces informations en vue d'un potentiel traitement. 
		\par
		On a donc décidé de récupérer ces informations dans une base de données qui serait gérée par une plateforme Web.
		\subsubsection{La base de données}
		\par
		La base de données que nous avons utilisée est celle offerte par Polytech'Lille, comme à chaque étudiant de l'école. Nous avons donc opté pour une base MySQL avec le panneau d'administration PHPMyAdmin. Le choix de MySQL est simplement justifié par une meilleure connaissance et habitude des requêtes et commandes SQL, contrairement à PosgreSQL (bien qu'il y ait peu de différences).
		\subsubsection{La plateforme Web}
		\par
		L'accès à la plateforme se fait via l'url suivante : \href{http://smartsensorwifi.plil.net}{Smart Sensor Wifi.}
		\\%
		Le site n'est pas en mode sécurisé (https) pour une raison simple : le shield Wifi ne supporte pas l'https dans le firmware actuel (celui livré de base). C'est donc pour cela que nous avons choisi de le mettre en http simple (bien qu'il puisse être basculé en https très rapidement et simplement sans poser aucun souci de compatibilité).
		\par
		Au niveau langage de programmation, le site est en HTML et PHP. Nous avons choisi le PHP, plutôt que le Python ou autre, car le PHP était plus familier et offrait toutes les opérations nécessaires. De plus, une petite partie du site utilise du Javascript afin de rendre plus dynamique et interactif le site.
		\par
		Au niveau design, nous avons décidé d'utiliser deux frameworks libres : Ink et Chart.js. Le premier gère le design global du site et le second est utile à la création de graphiques. Nous avons décidé d'utiliser des framework afin de rendre le site plus propre et d'offrir à l'utilisateur un outil visuellement agréable à utiliser.
		\par
		Au niveau technique, la plateforme est un simple intermédiaire entre, d'une part les capteurs et la base de données, et d'autre part, les utilisateurs et la base de données. D'un côté, le site récupère les informations que les capteurs lui envoie et stocke ces données dans la base. De l'autre côté,  à travers différents boutons et champs, l'utilisateur peut récupérer les informations qu'il désire.
		
	\subsection{Présentation de la partie capteur}
		\subsubsection{Le hardware}
		\par
		A la fin du projet, nous avons réalisé un prototype fonctionnel de ce
		que pourrait être le capteur. Il est important de noter que nous n'avons
		pas cherché à intégrer la solution mais plutôt à la développer en terme
		de fonctionnalités. De ce fait le prototype est assez volumineux mais
		sa taille peut aisément être réduite en utilisant des solutions à base
		de composants CMS notamment ce qui permettrait de réduire de manière
		significative l'encombrement de l'appareil.
		\par
		Le prototype final se compose donc de trois éléments principaux. Le coeur
		du système est réalisé avec un Arduino Uno. Une platine de développement
		sur microcontroleur basée sur un ATmega328P de chez Atmel. C'est dans cet
		Arduino qu'est flashé le programme du capteur. Vient en suite la partie
		qui prend en charge le WiFi, il s'agit d'un shield WiFi de chez GoTronic
		basé sur la puce WizFi210 de WizNet, cette dernière est interfacée avec
		l'Arduino via le port série. Enfin la dernière partie est un protoshield
		sur lequel est placé une breadboard sur laquelle sont enfichés les composants
		avec lesquels l'Arduino peut interagir, à savoir :
		\begin{itemize}
			\item Une photorésistante pour capter la luminosité
			\item Un capteur TMP36 de température
			\item Une RTC (Real Time Clock) DS1302
		\end{itemize}
		Notons que nous avons choisi ces capteurs pour leur facilité d'utilisation,
		mais dans l'absolu rien n'empêche d'utiliser n'importe quel autre capteur
		analogique ou TOR, ni même d'interfacer le système avec des choses plus
		complexes à base de SPI ou d'I$^2$C, étant donné que l'ATmega328P en a
		la possibilité.
		\par
		Le principe de fonctionnement du capteur est très simple. Une fois configuré
		à l'aide de \texttt{configurator} (dont le fonctionnement sera détaillé ci après)
		il vous suffit de déployer le capteur là où vous souhaitez qu'il opère et le
		laisser envoyer des mesures à interval d'une minute. On dispose ainsi
		d'un suivi efficace de l'environnement. Aussitôt que l'appareil est
		mis sous tension il va essayer de se connecter au réseau qui lui a été
		spécifié en configuration et une fois que cela s'est fait proprement il
		va passer en mode interruptif et enverra ses mises à jour toutes les minutes.
		Il faut aussi noter que pendant ce temps là, le capteur écoutera les connexions
		entrantes sur le port 80 et pourra ainsi être reconfiguré par ce biais à l'aide
		d'un set de commandes AT décris et documenté en annexe.
		
		\subsubsection{Le software}
		La partie logicielle du projet est découpée en deux parties. La première
		consiste en le firmware que nous avons chargé dans l'Arduino, et qui est chargé
		de gérer la carte WiFi et les différents capteurs.
		\par
		La seconde partie consiste en un logiciel de configuration écrit spécialement
		pour ce projet, il s'agit de \texttt{configurator}. C'est un script en python
		qui permet via une interface graphique plutôt intuitive de configurer un capteur,
		et ce que ça soit par le port série d'un capteur directement relié au PC en USB
		ou via le réseau en utilisant une socket TCP.
