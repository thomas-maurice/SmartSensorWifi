\section{Détail de la conception de la partie microcontrôleur du projet}
	\subsection{Le capteur}
		Lors de la réalisation du capteur il a fallu en premier lieu se fixer
		des objectifs à atteindre, afin d'être sur d'aller dans une direction
		cohérente tout a long du projet. De  ce fait nous avons choisi d'opter pour
		la réalisation d'un prototype fonctionnel et relativement simple au début.
		Par relativement simple, nous entendions capable de lire un ou deux capteurs
		analogiques simples à mettre en place de manière à nous focaliser sur
		l'interfaçage du dispositif avec un serveur de données. Nous nous sommes
		donc d'avantage focalisés sur une approche plus logicielle du projet en
		écartant volontairement, par manque de temps, des considérations annexes
		comme la gestion de l'énergie ou la mise en place immédiate de capteurs
		très complexes. Le but était réellement de fournir une base saine sur laquelle
		il serait facile de se baser à l'avenir.
		
		\subsubsection{La partie électronique}
		
		La partie électronique est la plus simple de l'appareil c'est donc celle
		là qui sera abordée en premier. Elle se compose simplement d'une photo résistance
		montée en pull up sur la patte analogique 1 de l'Arduino et d'un capteur de
		température câblé sur la patte analogique 2. Nous avons également câblé un bouton de
		mise à jour (dont l'utilité sera détaillée plus tard) sur la patte PD2
		ainsi que la real time clock sur trois pattes numériques de manière à pouvoir
		échanger des données. Tout ces composants ont été positionnés sur une breadboard
		de manière à ce que nous puissions facilement en modifier l'arrangement au fur et a mesure
		que nos besoins évoluaient dans le projet.
		\par
		Nous n'avons malheureusement pas eu le temps de réaliser de carte
		pour fixer ce design. Et il n'aurait pas été pertinent de le faire dans la mesure
		ou la conception et la fabrication d'une carte sont des activités pour
		le moins chronophages et qu'il a effectivement été plus productif pour nous de nous
		concentrer sur l'ajout de nouvelles fonctionalités au capteur. Cependant
		nous avons édités des PCB et des schémas électroniques disponibles en
		annexe qui permettent de réaliser une telle carte.
		
		\subsubsection{Conception du firmware de l'Arduino}
		
		
		\subsubsection{Principales difficultés rencontrées}
		
	\subsection{Configurator.py}
