\section{Détail de la conception de la partie microcontrôleur du projet}
	\subsection{Le capteur}
		Lors de la réalisation du capteur il a fallu en premier lieu se fixer
		des objectifs à atteindre, afin d'être sur d'aller dans une direction
		cohérente tout a long du projet. De  ce fait nous avons choisi d'opter pour
		la réalisation d'un prototype fonctionnel et relativement simple au début.
		Par relativement simple, nous entendions capable de lire un ou deux capteurs
		analogiques simples à mettre en place de manière à nous focaliser sur
		l'interfaçage du dispositif avec un serveur de données. Nous nous sommes
		donc d'avantage focalisés sur une approche plus logicielle du projet en
		écartant volontairement, par manque de temps, des considérations annexes
		comme la gestion de l'énergie ou la mise en place immédiate de capteurs
		très complexes. Le but était réellement de fournir une base saine sur laquelle
		il serait facile de se baser à l'avenir.
		\par
	
	\subsection{Configurator.py}
